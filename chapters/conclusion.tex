\begin{spacing}{1.5}
\phantom
\phantom
\phantom
\phantom
\section{Conclusion}
\phantom
\phantom
The challenge this research sought to address was to improve upon previous studies aimed at measuring the effectiveness of forest fire suppression in Ontario. Epistemological limitations aside, the only parsimonious conclusion to be made from this study, is that that the aggressive fire suppression strategy employed by Ontario's Ministry of Natural Resources--designed to limit the number of large forest fires, has indeed been effective over the period 1989--2004. This research has therefore, vindicated previous studies, which also found fire suppression to have significantly reduced the annual area burned by large wildfires, and is consistent with findings outside of Ontario, that showed fire suppression to reduce both the number and frequency of large fires in the forests of British Columbia and Alberta.\\

\noindent The study involved comparing the fire size distributions between several forested areas with contrasting fire management strategies, selected to control for other significant causal factors. Generalised Liner Models were used to calculate precisely the extent to which a more aggressive fire suppression strategy has effectively reduced the damage caused by wildfires over recent decades. \\

\noindent This new data on fire size distributions helps fire managers estimate how the forests in other fire management zones may react to changing fire suppression activities, thereby helping fire managers make more informed decisions about the province's forest management strategy. It will now also be possible to quantify the value of those resources protected by the OMNR's fire management strategy. Given that the OMNR expends a great deal of resources suppressing these wildfires, proving the fire management strategy employed, has delivered in its objectives, will go a long way to protecting those services against calls to reallocate funding and resources.\\

\clearpage

\noindent The methods used to derive these results may also be of interest to researchers using quantitative methods, as special consideration was given to ensuring the reproducibility of all calculations. The analysis was developed using the open-source R system for statistical computation and the code used in the analysis is include in the digital version of this file. These steps will therefore, allow for full replication, which is considered the gold standard in open Science. To aid with the dissemination of this resource, the R code of the project will be uploaded to the author's github repository (\texttt{https://github.com/craigrshenton/masters\_\linebreak thesis}) and distributed under the creative commons licence.\\


\end{spacing}
\clearpage
