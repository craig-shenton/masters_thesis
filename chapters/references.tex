\begin{spacing}{1.5}
\phantom
\phantom
\phantom
\phantom
\section{References}
\phantom
\phantom
\noindent Armstrong, G.W. 1999. A stochastic characterisation of the natural disturbance regime of the boreal mixed--wood forest with implications for sustainable forest management. \emph{Canadian Journal of Forest Research}, 29, 424--433.\\

\noindent Allen, C.D., Savage, M., Falk, D.A., Suckling, K.F., Swetnam, T.W., Schulke, T., Stacey, P.B., Morgan, P., Hoffman, M., and Klinglel, J.T. 2002. Ecological restoration of southwestern ponderosa pine ecosystems: A broad perspective. \emph{Ecological Applications}, 12(5), 1418--1433.\\

\noindent Allen, C.D. 2007. Interactions Across Spatial Scales among Forest Dieback, Fire, and Erosion in Northern New Mexico Landscapes. \emph{Ecosystems}, 10(5), 797--808.\\

\noindent Ashiq, M.W. 2011. Ontario boreal fire regimes in the context of lightning--caused ignition point spatial patterns. M.Sc. Thesis. Geography, University of Waterloo, Waterloo, Canada. [Online] Available at: http://uwspace.uwaterloo.ca/han\linebreak dle/10012 /6257. [Accessed 12 July 2012].\\

\noindent Bergeron, Y., Flannigan, M., Gauthier, S., Leduc, A., and Lefort, P. 2004. Past, current and future fire frequency in the Canadian boreal forest: implications for sustainable forest management. \emph{Ambio}, 33, 356--360.\\

\noindent Beverly, J. L., and Martell, D. L. (2005). Characterizing extreme fire and weather events in the boreal shield ecozone of Ontario. \emph{Agricultural and Forest Meteorology}, 133, 5--16.\\

\noindent Bonnicksen, T.M. 2002. Oversight Hearing on Wildfire on the National Forests. Committee on Forests and Forest Health, United States House of Representatives. [Online] Available at: http://naturalresources.house. gov/uploadedfiles/\linebreak thomas--bonnicksen--testimony--7.11.02.pdf. [Accessed 21 July 2012].\\

\noindent Bridge, S.R.J., Miyanishi, K. and Johnson, E.A. 2005. A Critical Evaluation of Fire Suppression Effects in the Boreal Forest of Ontario. \emph{Forest Science}, 51(1), 41--50.\\

\noindent Calkin, D.E., Gebert, K.M., Jones, J.C., and Neilson, R.P. 2005. Forest Service large fire area burned and suppression expenditure trends, 1970--2002. \emph{Journal of Forestry}, 103, 179--183.\\

\noindent Christensen, N.L., Agee, J.K., Brussard, P.F., Hughes, J., Knight, D.H., Minshall, G.W., Peek, J.M., Pyne, S.J., Swanson, F.J., Thomas, J.W., Wells, S., Williams S.E., and Wright, H.A. 1989. Interpreting the Yellowstone fires of 1988. \emph{Bioscience}, 39, 678--685.\\

\noindent Collett, D. 1991. \emph{Modelling binary data}. London: Chapman and Hall.\\

\noindent Croarkin, C. and Tobias, P. 2003. Engineering Statistics Handbook: Chapter 1.3.6.7.3 Upper Critical Values of the F Distribution. [Online] Available at: http://w ww.itl.nist.gov/div898/handbook/eda/section3/eda3673.htmONE--01--1--10. [Accessed 21 June 2012].\\

\noindent Cui, W. and Perera, A.H. 2008. What do we know about forest fire size distribution, and why is this knowledge useful for forest management? \emph{International Journal of Wildland Fire}, 17, 234--244.\\

\noindent Cumming, S.G. 2001. A parametric model of the fire--size distribution. \emph{Canadian Journal of Forest Research}, 31, 1297--1303.\\

\noindent Cumming, S.G. 2005. Effective fire suppression in boreal forests. \emph{Canadian Journal of Forest Research}, 35(4), 772--786.\\

\noindent Cunningham, A.A., and Martell, D.L. 1976. The use of subjective probability assessments to predict forest fire occurrence. \emph{Canadian Journal of Forest Research}, 6, 348--356.\\

\noindent DiBari, J.N. 2003. Scaling exponents and rank--size distributions as indicators of landscape character and change. \emph{Ecological Indicators}, 3, 275--284.\\

\noindent Dobson, A.J. 2002. \emph{An Introduction to Generalized Liner Models}. London: Chapman and Hall.\\

\noindent Donovan, G.H. and Noordijk, P. 2005. Assessing the accuracy of wildland fire situation analysis (WFSA) fire size and suppression cost estimates. \emph{Journal of Forestry}, 105, 10--13.\\

\noindent Donovan, G.H., and Brown, T.C. 2007. Be careful what you wish for: The legacy of Smokey Bear. \emph{Frontiers in Ecology and the Environment}, 5(2), 73--79.\\

\noindent Drushka, K. 2003. \emph{Canada's forests: a history}. Montreal: McGill Queen's University Press.\\

\noindent Finney, D.J. 1973. \emph{Statistical Methods in Bioassay}. New York: Hafner.\\

\noindent Flannigan, M.D., Krawchuk, M.A., de Groot, W.J., Wotton, B.M. and Gowman, L.M. 2009. Implications for changing climate for global wildland fire. \emph{International Journal of Wildland Fire}, 18, 483--507.\\

\noindent Franklin, J.F. 1993. Fueling controversy and providing direction: lessons from old growth. \emph{Journal of Forestry}, 91, 11--13.\\

\noindent Friedrich Leisch, F. 2002. \emph{Dynamic generation of statistical reports using literate data analysis}. In: "Compstat 2002 -- Proceedings in Computational Statistics". (Eds) Hardle, W. and Ronz, B. Heidelberg: Physica Verlag, 575--580. \\

\noindent Gauthier, S., Valliancourt, M.A., Kneeshaw, D., Drapeau, P., Grandpre, L.D., Claveau, Y. and Pare, D. 2009. \emph{Forest ecosystem management: origins and foundations}. In Gauthier, S., Valliancourt, M.A., Leduc, A., Grandpre, L.D., Kneeshaw, D., Morin, H., Drapeau, P. and Bergeron, Y. (eds.), "Ecosystem management in the boreal forests". Quebec: Presses de l'Universite  du  Quebec.\\

\noindent Gentleman, R. and Lang, D.T. 2007. Statistical Analyses and Reproducible Research. \emph{Journal of Computational and Graphical Statistics}, 16(1), 1--23.\\

\clearpage

\noindent Gill, J. 1999. The Insignificance of Null Hypothesis Significance Testing. \emph{Political Research Quarterly}, 9, 1--32.\\

\noindent Hawkes, B., Vasbinder, W., and DeLong, G. 1997. Retrospective Fire Study--Fire Regimes in the SBSvk \& ESSFwk2/wc3 Biogeoclimatic Units of North--Eastern British Columbia. McGregor Model Forest Network: Prince George, BC.\\

\noindent Haydon, D.T., Friar, J.K., Pianka, E.R. 2000. Fire--driven dynamic mosaics in the Great Victoria Desert, Australia. \emph{Landscape Ecology,} 15, 373--382.\\

\noindent Heyerdahl, E.K., Brubaker, L.B., and Agee, J.K. 2001. Spatial controls of historical fire regimes: a multiscale example from the interior west, USA. \emph{Ecology}, 82, 660--678.\\

\noindent Hills, G. A. (1959). A ready reference to the description of the land of Ontario and its productivity. Toronto, ON: Ontario Department of Land and Forests.\\

\noindent Hills, G.A. (1969). A physiographic approach to the classification of terrestrial ecosystems with respect to representative biological communities. Toronto, ON: Ontario Department of Land and Forests.\\

\noindent Hirsch, K.G., Corey, P.N., and Martell, D.L. 1998. Using expert judgment to model initial attack fire crew effectiveness. \emph{Forest Science}, 44, 539--549.\\

\noindent Hirotugu, A. 1974. A new look at the statistical model identification. \emph{IEEE Transactions on Automatic Control}, 19(6), 716--723.\\

\noindent Holling, C.S. and Meffe, G.K. 1996. Command and Control and the Pathology of Natural Resource Management. \emph{Conservation Biology,} 10(2), 328--337.\\

\noindent Johnson, E.A., Miyanishi, K. and Bridge, S.R.J., 2001. Wildfire Regime in the Boreal Forest and the Idea of Suppression and Fuel Buildup. \emph{Conservation Biology}, 15(6), 1554--1557.\\

\clearpage

\noindent Kilgore, B.M. 1976. Fire management in the natural parks: an overview. \emph{Proceedings of Tall Timbers fire ecology}. Tallahassee: Florida State University.\\

\noindent Latremouille, C., Parker, W. C., McPherson, S., Pinto, F., Fox, B., and McKinnon, L. (2008). Ecology and management of eastern white pine in the lake abitibi (3E) and lake temagami (4E) ecoregions of Ontario. Science Development and Transfer Series No. 004. Sault Ste Marie, Ontario: Ontario Forest Research Institute, Ontario Ministry of Natural Resources.\\

\noindent Leisch, F. 2002. Sweave: Dynamic generation of statistical reports using literate data analysis. In Hardle, W. and Ronz, B. (eds), \emph{Compstat 2002 -- Proceedings in Computational Statistics}, 575--580.\\

\noindent Makridakis, S. and Taleb, N. 2009. Decision making and planning under low levels of predictability. \emph{International Journal of Forecasting}, 5, 1--18.\\

\noindent Malamud, B.D., Morein, G., and Turcotte, D.L. 1998. Forest fires: an example of self--organized critical behavior. \emph{Science}, 281, 1840--1842.\\

\noindent Martell, D.L. 1994. The impact of fire on timber supply in Ontario. \emph{The Forestry Chronicle}, 70, 164--173.\\

\noindent Martell, D.L., Kourtz, P.H., Tithecott, A., and Ward, P.C. 1999. The Development and Implementation of Forest Fire Management Decision Support Systems in Ontario, Canada. USDA Forest Service General Technical Report PSW--GTR--173, 131--142.\\

\noindent Martell, D.L. 2001. 'Forest fire management'. In \emph{Forest Fires: behavior and ecological effects}. E.A. Johnson and K. Miyanishi (Eds). San Diego, California: Academic Press, 572--583.\\

\noindent Martell, D.L. 2002. Wildfire regime in the boreal forest. \emph{Conservation Biology}, 16(5), 1177--1178.\\

\noindent McCullagh, P., and Nelder, J.A. 1983.\emph{ Generalised linear models}. London: Chapman and Hall.\\

\noindent Merrill, D.F., and Alexander, M.E. 1987. \emph{Glossary of forest fire management terms}. 4th Edition. Canadian Committee on Forest Fire Management, National Research Council of Canada, Ottawa.\\

\noindent Minnich, R.A. 1983. Fire mosaics in Southern California and Northern Baja California. \emph{Science}, 219, 1287--1294.\\

\noindent Minnich, R.A., Chou, Y.H. 1997. Wildland fire patch dynamics in the chaparral of southern California and northern Baja California. International Journal of Wildland Fire, 7, 221--248.\\

\noindent Miyanishi, K., and Johnson, E.A. 2001. A re--examination of the effects of fire suppression in the boreal forest. \emph{Canadian Journal of Forest Research}, 31, 1462--1466.\\

\noindent Miyanishi, K., Bridge, S.R.J., and Johnson, E.A. 2002. Wildfire regime in the boreal forest. \emph{Conservation Biology}, 16(5), 1177--1178.\\

\noindent Moreno, J.M., Vazquez, A., and Velez, R. 1998. \emph{Recent history of forest fires in Spain}. In 'Large Forest Fires'.  Leiden: Backhuys Publishers.\\

\noindent Murphy, P.J., Mudd, B.J. Stocks, E.S., Kasischke, D., Barry, M.E., Alexander, N.F.H., and French, N. 2000. \emph{Historical fire records in the North American boreal forest}. In "Fire, Climate Change, and Carbon Cycling in the Boreal Forest", \emph{Ecological Studies}, 138, 274--288.\\

\noindent Nelder, J. and Wedderburn, R. 1972. \emph{Generalized Linear Models}. \emph{Journal of the Royal Statistical Society}. Series A (General), 135(3), 370--384.\\

\clearpage

\noindent Ontario Ministry of Natural Resources / Ministere des Richesses Naturelles (OMNR). 1989. \emph{Guidelines for modifying woods operations}. Aviation, Flood, and Fire Management Branch, Ontario Ministry of Natural Resources. Toronto, Canada. \\

\noindent Ontario Ministry of Natural Resources / Ministere des Richesses Naturelles (OMNR). 1997. \emph{Ontario's fire management strategies-- Background, current state, and future challenges}. Toronto, Canada.\\

\noindent Ontario Ministry of Natural Resources / Ministere des Richesses Naturelles (OMNR). 2001. \emph{Ontario's  forests  fact  sheet: The boreal forests}. [Online] Available at: http://www.mnr.gov.on.ca/stdprodconsume/groups/lr/@mnr/@forests/\linebreak documents/document/241209.pdf. [Accessed 12 Apr 2012].\\

\noindent Ontario Ministry of Natural Resources / Ministere des Richesses Naturelles (OMNR). 2007. \emph{Ecological Land Classification Primer}. Toronto, Canada. [Online] Available at: http://www.mnr.gov.on.ca/en/Business/LUEPS/Publication\linebreak /264779.html. [Accessed 12 July 2012].\\

\noindent Papoulis, A. 1984. "Bernoulli Trials." Chapter 3.2. In \emph{Probability, Random Variables, and Stochastic Processes.} New York: McGraw-Hill.\\

\noindent Peng, R.D. 2011. Reproducible research in computational science. \emph{Science}, 334, 1226--1227.\\

\noindent Perera, A. H., Baldwin, D. J. B., Schnekenburger, F., Osborne, J. E., and Bae, R. E. (1998). Forest fires in Ontario: a spatio--temporal perspective. \emph{Forest Research Report No. 147}. Sault Ste Marie, ON. Ontario Forest Research Institute, Ontario Ministry of Natural Resources.\\

\noindent Perera, A. H., Cui, W., and Ouellette, M. R. (2009). Size class distribution and spatial proximity of fires  in  a  simulated  boreal  forest  fire  regime  in  relation  to  Ontario's  policy  directions  for  emulating   natural disturbance. \emph{Forest Research Report No. 170}. Sault Ste Marie, ON: Ontario Forest Research Institute, Ontario Ministry of Natural Resources.\\

\noindent Podur, J., Martell, D. L., and Csillag, F. (2003). Spatial patterns of lightning--caused forest fires in ontario, 1976--1998.\emph{ Ecological Modelling}, 164, 1--20.\\

\noindent R Development Core Team. 2004. R: a language and environment for statistical computing. R Foundation for Statistical Computing, Vienna, Austria. \\

\noindent R--Studio Development Team. 2011. R--Studio Integrated Development Environment (Version 2.13.0). [Computer program] Available at: http://www.rstudio\linebreak .org/. [Accessed 09 Apr 2012].\\

\noindent Racey, G. D., Wiltshire, R. O., and Archibald, D. J. (2000). Ecoregional forest composition analysis for northwestern Ontario: Present forest condition. \emph{NWST Technical Report TR--123}. Thunder Bay, ON: Northwest Science and Technology Ontario Ministry of Natural Resources.\\

\noindent Schwab, M., Karrenbach, M. and Claerbout, J. 2000. Making scientific computations reproducible. \emph{Computing in Science and Engineering}, 2.6, 61--67.\\

\noindent Snedecor, G.W. and Cochran, W.G. 1967. \emph{Statistical Methods} (6th ed.). Iowa: Iowa State University Press.\\

\noindent Song. W., Fan. W., Wang. B., and Zhou. J. 2001. Self--organized criticality of forest fires in China. \emph{Ecological Modeling}, 145, 61--68.\\

\noindent Stocks, B.J. 1991. "The extent and impact of forest fires in northern circumpolar countries." In, \emph{Global biomass burning: atmospheric, climatic and biospheric implications}. Levine, J.S. (ed.). Cambridge, MA: MIT Press.\\

\noindent Stocks, B.J., Mason, J.A., Todd, J.B., Bosch, E.M., Wotton, B.M., and Amiro B.D. 2002. Large forest fires. \emph{Canadian Journal of Geophysysical Research}, 107, 8149.\\

\noindent Strauss, D., Bednar, L., and Mees, R. 1989. Do one percent of forest fires cause ninety--nine percent of the damage? \emph{Forest Science}, 35(2), 319--328.\\

\clearpage

\noindent Suffling, R. (1995). Can disturbance determine vegetation distribution during climate warming? A boreal test. \emph{Journal of Biogeography}, 22, 501--508.\\

\noindent Taleb, N.N. 2007. Black Swans and the Domains of Statistics. \emph{The American Statistician}, 61(3), 1--3.\\

\noindent Taleb, N.N. 2008. The Fourth Quadrant: A Map of the Limits of Statistics. \emph{Edge}. [Online] Available at: http://www.edge.org/3rd--culture/taleb08/taleb08--index.html. [Accessed 22 July 2012]. \\

\noindent Telesca, L., Amatulli, G., Lasaponara, R., Lovallo, M., and Santulli, A 2005. Time--scaling properties in forest--fire sequences observed in Gargano area (southern Italy). \emph{Ecological Modeling}, 185, 531--544.\\

\noindent Turner, R. (2009). Point patterns of forest fire locations. \emph{Environmental and Ecological Statistics}, 16, 197--223.\\

\noindent Van Wagner, C.E. 1978. Age--class distribution and the forest fire cycle. \emph{Canadian Journal of Forest Research}, 8, 220--227.\\

\noindent Vazquez, A. and Moreno, J.M. 2001. Spatial distribution of forest fires in Sierra de Gredos (Central Spain). \emph{Forest Ecology and Management}, 147, 55--65.\\

\noindent Vitek, J. and Kalibera, T. 2011. Repeatability, Reproducibility and Rigor in Systems Research. \emph{EMSOFT '11}, 33--38.\\

\noindent Ward, P.C., and Tithecott, A.G. 1993. \emph{Impact of fire management on the boreal landscape of Ontario.} Ministry of Natural Resources, Aviation, Flood and Fire Management Branch, Sault Sainte, Marie, Ontario.\\

\noindent Ward, P.C., Tithecott, A.G., and Wotton, B.M. 2001. Reply--A re--examination of the effects of fire suppression in the boreal forest. \emph{Canadian Journal of Forest Research}, 31, 1467--1480.\\

\noindent Weber, M.G., and Stocks, B.J. 1998. \emph{Forest fires in the boreal forests of Canada}. In, "Large forest fires." Leiden, Netherlands: Blackhuys Publishers.\\

\noindent Weir, J.M.H., Johnson, E.A., and Miyanishi, K. 2000. Fire frequency and the spatial age mosaic of the mixed--wood boreal forest in western Canada. \emph{Ecological Applications}, 10, 1162--1177.\\
\end{spacing}
\clearpage
