\begin{spacing}{1.5}
\phantom
\phantom
\phantom
\phantom
\section*{Abstract}
\phantom
\phantom
A great deal of effort and resources are expended in suppressing wildfires in the boreal forests of Ontario, Canada. Fire managers at Ontario's Ministry of Natural Resources had assumed these efforts to be effective, given that data from Ontario showed fire suppression had significantly reduced the annual area burned by wildfires over recent decades. However, fire--ecologists have brought to light flaws in the fire manager's analysis, making their assumption of effectiveness unsound. \\

\noindent The challenge this research sought to address therefore, was to improve upon these past efforts at measuring the effectiveness of forest fire suppression in Ontario, whilst maintaining a scientific integrity in both the research design and analysis. The study involves comparing the fire size distributions between several areas with contrasting forest fire management strategies, selected to control for other significant causal factors. With this method, it is possible to calculate precisely the extent to which a more aggressive fire suppression strategy has effectively reduced the damage caused by wildfires over recent decades. \\




\end{spacing}
\clearpage
