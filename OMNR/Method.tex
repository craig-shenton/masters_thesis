\documentclass{article}
\usepackage{setspace} 
\usepackage{graphicx}
\addtolength{\topmargin}{-.575in}
\addtolength{\textheight}{1.25in}
\renewcommand{\familydefault}{\sfdefault}


\usepackage{Sweave}
\begin{document}
\Sconcordance{concordance:Method.tex:Method.Rnw:%
1 8 1 1 0 26 1 1 2 4 0 1 2 2 1 1 4 2 1 1 6 2 1 1 12 2 1 1 14 1 1 1 4 6 %
1 1 2 1 0 1 1 1 3 1 0 1 2 29 0 1 3 1 1 1 5 16 0 1 2 3 1}


\begin{spacing}{1.5}
\phantom
\phantom
\phantom
\phantom
\section*{Chapter 4 \\ \huge{Methodology}}
\phantom
\phantom
\subsection*{4.2 Statistical Methods}

From provincial fire management records in the Ontario Ministry of Natural Resources (OMNR) fire database the following annual statistics will be calculated:\\

\begin{tabular}{ll}
$E_{\mathrm{t}}$     & the annual number of escapes (i.e., fires  $\geq$3 {\it ha})\\
$L_{\mathrm{t}}$     & the annual number of large fires (i.e., fires $\geq$200 {\it ha})\\
\end{tabular}\\

\noindent As Cumming (2005:775) suggests, the observed annual distribution of fires ($L_{\mathrm{t}}$ / $E_{\mathrm{t}}$) will be used as an estimate of the annual probability ($\rho{\mathrm{t}}$) that a randomly chosen escaped fire ($E$) will fail to be suppressed and become a large fire ($L$). \\

Conditional on $E_{\mathrm{t}}$ (i.e., the annual number of escapes), $L_{\mathrm{t}}$ (i.e., the annual number of large fires) will, therefore, be a {\it binomial random variable} with an expected value of $L_{\mathrm{t}}$ $\rho{\mathrm{t}}$ and variance $L_{\mathrm{t}}$ $\rho{\mathrm{t}}$ (1 $-$ $\rho{\mathrm{t}}$). As such, the hypotheses $H_{\mathrm{0}}$ and $H_{\mathrm{S}}$ can be operationalised as logistic regression models of the annual variation in $\rho{\mathrm{t}}$, and tested by regressing the observations against the suppression strategy ($S_{\mathrm{i}}$ or $S_{\mathrm{m}}$) being employed. \\


\noindent With the OMNR dataset .csv file downloaded to the working directory, the file was loaded into R.


\noindent To calculate $N_{\mathrm{t}}$ the annual load, the total number of fires per year were aggregated.


\noindent The OMNR dataset was then split into the two Fire Management Zones, Inensive and Measured, reqired for the analysis.


\noindent The OMNR dataset was further seperated into the number of suppressed and escaped fires in each Fire Management Zone.


\noindent $N_{\mathrm{s}}$ and $N_{\mathrm{e}}$ could then aggregated.



\begin{Schunk}
\begin{Sinput}
> omnr.glm <- data.frame(zone=rep(c("mea","int"),c(16,16)),year=rep(1989:2004,2),suppressed=c(49,10,52,28,5,34,58,33,21,23,12,6,6,11,10,1,137,83,12,79,42,93,199,89,73,150,83,57,101,38,77,16),unsuppressed=c(7,0,14,2,2,4,26,27,6,4,6,0,2,6,6,0,7,5,8,7,1,1,27,54,5,15,12,1,0,10,10,1),load=c(2429,1614,2559,960,743,1053,2122,1245,1634,2278,1017,644,1561,1138,1036,428,2429,1614,2559,960,743,1053,2122,1245,1634,2278,1017,644,1561,1138,1036,428))
\end{Sinput}
\end{Schunk}

\begin{Schunk}
\begin{Soutput}
Call:
glm(formula = glm.eq, family = binomial(logit), data = omnr.glm)

Deviance Residuals: 
    Min       1Q   Median       3Q      Max  
-8.3516  -0.3858   0.6594   1.8976   5.0376  

Coefficients:
                                Estimate Std. Error z value Pr(>|z|)
(Intercept)                     35.03499   40.31208   0.869   0.3848
omnr.glm$zonemea               146.85587   70.22388   2.091   0.0365
omnr.glm$year                   -0.01650    0.02019  -0.817   0.4138
omnr.glm$zonemea:omnr.glm$year  -0.07409    0.03519  -2.105   0.0353

(Dispersion parameter for binomial family taken to be 1)

    Null deviance: 238.46  on 31  degrees of freedom
Residual deviance: 183.76  on 28  degrees of freedom
AIC: 288.11

Number of Fisher Scoring iterations: 5
\end{Soutput}
\end{Schunk}

\begin{center}
% latex table generated in R 2.15.0 by xtable 1.7-0 package
% Mon Jul 23 16:08:40 2012
\begin{table}[ht]
\begin{center}
\begin{tabular}{rrrrr}
  \hline
 & Estimate & Std. Error & z value & Pr($>$$|$z$|$) \\ 
  \hline
(Intercept) & 35.0350 & 40.3121 & 0.87 & 0.3848 \\ 
  omnr.glm\$zonemea & 146.8559 & 70.2239 & 2.09 & 0.0365 \\ 
  omnr.glm\$year & -0.0165 & 0.0202 & -0.82 & 0.4138 \\ 
  omnr.glm\$zonemea:omnr.glm\$year & -0.0741 & 0.0352 & -2.11 & 0.0353 \\ 
   \hline
\end{tabular}
\caption{Summary statistics for the regression model}
\label{tab:summary}
\end{center}
\end{table}      \end{center}

\end{spacing}
\end{document}
